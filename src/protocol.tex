\chapter{Protocollo MDSP}

\label{cap:protocollo}

\section{Progettazione}


MDSP è l'acronimo di Multiple Description Stream Protocol, un protocollo
per lo scambio di stream di vario tipo (contenuti multimediali) organizzati secondo una
meta-codifica a descrizioni multiple.

\begin{figure}[hb]
\includegraphics[width=0.90\textwidth]{../images/network_mdc.png}
\label{fig:network_mdc}
\caption{Modello di rete basata su MDC}
\end{figure}

Tale protocollo è concepito per possedere le seguenti caratteristiche:
\begin{itemize}
  \item ogni messaggio è indipendente dagli altri
  \item nessuna ritrasmissione dei messaggi errati o mancanti
  \item nessun tipo di controllo di errore
  \item nessun tipo di controllo di flusso 
\end{itemize}

Le ultime tre caratteristiche, se necessarie, possono essere gestite a livelli
più alti dello stack protocollare.



\subsection{Canali}
Devono essere previsti ``canali'' per il trasporto di due tipi di informazioni:
\begin{itemize}
  \item \emph{messaggi di controllo}: per la gestione della rete
  \item \emph{dati}: i contenuti scambiati dagli utenti
\end{itemize}




\subsection{Protocollo di trasporto}
Considerando i requisiti, è stato scelto UDP come protocollo per il trasporto
dei messaggi.

\subsection{Messaggi}


Le componenti del sistema che si trovano su elaboratori differenti comunicano tra
loro per mezzo di 'messaggi'. Tali messaggi rappresentano una particolare
richiesta di servizi, oppure una corrispondente risposta.

Un messaggio possiede una 'tipologia' e un certo numero di 'argomenti', dipendenti dal tipo di messaggio.

Il tipico messaggio è composto da un header di 8 byte, cosi' suddivisi:



\begin{itemize}
\item 3 byte per la stringa MDC

\item 1 byte per la versione del protocollo

\item 4 byte per il tipo di messaggio
\end{itemize}

I messaggi possono avere una o più stringhe ``null-terminated'',
ognuna delle quali descrive uno o più parametri. I parametri hanno un
\emph{nome} e un \emph{valore}, e più parametri sono separati dal segno di
``;'' (punto e virgola). L'esempio seguente:

\begin{code}
n=nome;h=12;
\end{code}

mostra due parametri: \emph{n} che ha valore ``\emph{nome}'' e \emph{h} con
valore \emph{12}.

Vedremo nelle sezione seguenti i principali messaggi e i loro parametri.




\subsubsection*{LIST}
%
\emph{LIST} è un messaggio utilizzato quando di vuole conoscere la lista degli
stream multimediali presenti su un certo host.

Parametri:
\begin{itemize}
  \item \emph{n}: stringa che indica il nome o un suo frammento, usata per
  circoscrivere la ricerca
\end{itemize}

\subsubsection*{ALST}
%
\emph{ALST} (contrazione di ``Answer LiST'') è la tipica risposta ad un
messaggio LIST. Il primo byte del contenuto rappresenta il numero di stream
trovati, ed è seguito da stringhe null-terminated contenenti i seguenti
parametri:
\begin{itemize}
  \item \emph{n}: stringa con il nome completo dello stream
  \item \emph{h}: stringa di 32 byte che rappresenta lo \emph{stream id}
  (univoco per tutto il sistema)
\end{itemize}

\subsubsection*{SINF}
\emph{SINF} (contrazione di ``Stream INFormation'') è usato per richiedere
informazioni su un determinato stream multimediale, come il numero di
descrizioni in cui è suddiviso e il numero di descrittori.

Parametro:
\begin{itemize}
  \item \emph{h}: \emph{stream id} dello stream richiesto
\end{itemize}

\subsubsection*{ASNF}
%
\emph{ASNF} (contrazione di ``Answer Stream iNFormation'') è generato in
risposta ad un messaggio SINF.

Parametri:
\begin{itemize}
  \item \emph{h}: \emph{stream id} dello stream multimediale richiesto
  \item \emph{fn}: \emph{flows numebr}, ossia il numero totale di
  \emph{descrizioni} (sotto-flussi MDC)
  \item \emph{sn}: \emph{sequence number}, ossia il numero totale di descrittori
\end{itemize}


\subsubsection*{SREQ}
%
\emph{SREQ} (contrazione di ``Stream REQuest'') viene utilizzato per richiedere
determinati descrittori di un particolare stream.

Parametri:
\begin{itemize}
  \item \emph{h}: \emph{stream id} dello stream multimediale richiesto
  \item \emph{f}: \emph{flow id}, ossia identificativo della particolare
  \emph{descrizione} (sotto-flusso)
  \item \emph{sb}: \emph{sequence begin}, ossia il numero del primo descrittore
  della sequenza richiesta
  \item \emph{se}: \emph{sequence end}, ossia il numero dell'ultimo descrittore
  della sequenza richiesta
\end{itemize}


\subsubsection*{ASRQ}
%
\emph{ASRQ} (contrazione di ``Answer Stream ReQuest'') viene utilizzato in
risposta al messaggio SREQ, come conferma dell'inizio dell'invio dei
descrittori scelti. 

Parametri:
\begin{itemize}
  \item \emph{h}: \emph{stream id} dello stream multimediale richiesto
  \item \emph{f}: \emph{flow id}, ossia identificativo della particolare
  \emph{descrizione} (sotto-flusso)
  \item \emph{sb}: \emph{sequence begin}, ossia il numero del primo descrittore
  della sequenza richiesta
  \item \emph{se}: \emph{sequence end}, ossia il numero dell'ultimo descrittore
  della sequenza richiesta
\end{itemize}

Contiene gli stessi parametri di SREQ, ma i valori sono
modificati in relazione allo stato dell'host. Per esempio, se vengono chiesti i
descrittori in una sequenza da 0 a 30, mentre l'host dispone soltanto dei
descrittori da 4 a 10, allora ASRQ conterrà $sb=4$ e $se=10$.

Subito dopo l'invio di un messaggio ASRQ l'host inizierà ad inviare i
descrittori richiesti sul canale dati.

\subsubsection*{PEER}
\emph{PEER} è usato per richiedere
informazioni su tutti i peer conosciuti da un peer.

Nessun parametro.


\subsubsection*{APER}
%
\emph{APER} (contrazione di ``Answer PEeR'') è generato in
risposta ad un messaggio PEER. Il primo byte del contenuto rappresenta il
numero di peer conosciuti, ed è seguito da stringhe null-terminated contenenti i
seguenti parametri:
\begin{itemize}
  \item \emph{a}: ip address del peer
  \item \emph{p}: porta del peer
\end{itemize}



\subsection{Descrittore}


Un \emph{Descriptor} è, nell'ambito di questo sistema, una unità di dati di un
flusso multimediale.

Il flusso multimediale è suddiviso in sotto-flussi detti \emph{descrizioni},
secondo le specifiche della codifica MD; ogni descrizione è composta da un certo numero
di descrittori, con la stessa relazione che c'è tra un filmato e un fotogramma.

Il Descriptor contiene varie informazioni, come il flusso originario, il numero
di descrizione, e la sequenza temporale. Il payload non è definito e pertanto è
permessa una maggiore libertà in fase di definizione dei codec (vedasi
paragrafo \ref{sec:esempio_codec}).

\subsection{Casi d'uso}

Chiameremo Alice un generico nodo della rete che vuole effettuare le azioni, e
Bob il peer a cui queste comunicazioni sono dirette.

Ci sono poi anche altri amici, come Carlina, David ed Emy, che condividono le
stesse passioni di Alice e Bob.


\subsubsection{Richiesta di un flusso multimediale}
%

Alice ha bisogno di un flusso multimediale, e sa già che Bob ce l'ha, tutto o in
parte (lo sa perché ha già visto la lista dei flussi di Bob).

Quindi Alice invia un messaggio SREQ, specificando come parametro l'hash univoco
del flusso, il numero della descrizione, e una sequenza di descrittori.

Bob quindi invierà un messaggio ASRQ, specificando come parametro il timeout di
keep-alive; Bob, infatti, continuerà ad inviare il flusso di dati finché riceverà
pacchetti di KALV (keep-alive) da Alice, supponendola disconnessa allo scadere
del timeout.






\subsubsection{Richiesta della lista dei flussi}
%

\begin{itemize}
\item Alice ha bisogno della lista dei flussi disponibili da Bob. Invia un messaggio LIST, non parametro vuoto, e riceve da Bob un messaggio ALST (Answer LiST) contenente un vettore con tutti i nomi e gli hash dei flussi che egli possiede;
\item Alice sta cercando un particolare flusso, di cui sa parte del nome
('pippo'); invia un messaggio LIST a Bob con parametro 'pippo', e Bob
risponderà con un ALST contenente informazioni su tutti i file che contengono la
parola 'pippo' nel nome.
\end{itemize}




\subsubsection{Richiesta di informazioni su un flusso}
%

Alice sa che un determinato flusso (di cui possiede l'hash) è posseduto da Bob;
invia quindi un messaggio SINF (Stream INFormation), per avere informazioni
specifiche del flusso, ad esempio la durata, la qualità di codifica, ecc. Bob
invia un messaggio ASNF contenente la risposta.


\subsubsection{Ricerca di altri peer}
Alice sta scaricando un flusso da Bob, ma avverte la necessità di frequentare
altre persone, che magari condividono lo stesso flusso; invia quindi un messaggio
PEER a Bob, seguito dall'hash univoco del flusso. Bob raccoglie le idee su tutti
gli amici che conosce, o che ha conosciuto tramite lo scaricamento del flusso in
oggetto, per esempio Carlina, da cui Bob ha scaricato originariamente il flusso,
David, che ha scaricato il flusso tempo fa, ed Emy che lo sta scaricando ora. Bob
quindi prepara un messaggio APER contenente l'indirizzo ip e la porta di Carlina,
David ed Emy e lo invia ad Alice.

