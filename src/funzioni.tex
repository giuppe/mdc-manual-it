\chapter{Funzioni}
\section{Funzione di codifica testuale}

La funzione di codifica testuale è contenuta nel file
\textit{text/md/codec.cpp} presente nella directory
\textit{mdc/src/codecs/text/}.

\begin{code}
void TextMDCodec::code(AbstractStream* stream, MDStream* md\_stream) const \{\\
	Uint32 stream\_size = stream->get\_data\_dim();\\
	Uint32 flow\_dimension = (stream\_size/m\_flows\_number)+1;\\
	Uint32 descriptors\_number = \\(Uint32)ceil(((double)flow\_dimension)/((double)m\_preferred\_payload\_size));\\
	Uint16 max\_payload\_size = (flow\_dimension/descriptors\_number)+1;\\
	md\_stream->init(stream->compute\_hash\_md5(), m\_flows\_number, descriptors\_number);\\
	for (Uint8 i=0; i<m\_flows\_number; i++) \{\\
		Uint64 offset = 0;\\
		for (Uint32 j=0; j<descriptors\_number; j++) \{\\
			if (stream\_size-(offset+i) > 0) \{\\
				Descriptor* descriptor = new Descriptor();\\
				descriptor->set\_stream\_id(md\_stream->get\_stream\_id());\\
				descriptor->set\_flow\_id(i);\\
				descriptor->set\_sequence\_number(j);\\
				descriptor->set\_codec\_name(string("text"));\\
				TextCodecParameters* tcp = new TextCodecParameters();\\
				descriptor->set\_codec\_parameter(tcp);\\
				MemDataChunk payload;\\
				Uint64 k;\\
				for (k=0; k<max\_payload\_size; k++)\\
					if (offset+i+k+m\_flows\_number < stream\_size)\\
						payload += \&stream->get\_data(offset+i+(k*m\_flows\_number), 1);\\
				offset += m\_flows\_number*k;\\
				descriptor->set\_payload(payload);\\
				md\_stream->set\_descriptor(descriptor);\\
			\}\\
		\}\\
	\}\\
\}\\
\end{code}

\begin{notabene}
La dicitura \textit{flow} viene usata con il significato di
`\emph{Descrizioni}'. Tale variazione rispetto alla nomenclatura presente in
letteratura è stata adottata al fine di evitare la possibile confusione dovuta
alla somiglianza dei termini `\emph{Descrizioni}' e `\emph{Descrittori}'. In seguito, quindi, la parola `\emph{Flussi}' è da intendersi con il significato di `\emph{Descrizioni}'.
\end{notabene}

\section{Funzione di decodifica testuale}
\label{cap:decode}
La funzione di decodifica testuale è contenuta nel file
\textit{text/md/codec.cpp} presente nella directory
\textit{mdc/src/codecs/text/}.

\begin{code}
void TextMDCodec::decode(const MDStream* md\_stream, AbstractStream* stream) const \{\\
	if (md\_stream->is\_empty()) \\
		return;\\
	LOG\_INFO("Decoding...");\\
	MemDataChunk* dc = new MemDataChunk();\\
	vector<Uint8> taken\_stream;\\
	Uint8 flows\_number = md\_stream->get\_flows\_number();\\
	LOG\_INFO("Flows number: "$<<$flows\_number);\\
	Uint32 sequences\_number = md\_stream->get\_sequences\_number();\\
	LOG\_INFO("Sequences number: "$<<$sequences\_number);\\
	Uint64 max\_dimension = 0;\\
	for (Uint8 i=0; i<flows\_number; i++)\{
		Uint64 offset = 0;\\
		Uint16 payload\_size = 0;\\
		for (Uint32 j=0; j<sequences\_number; j++)\{
			Descriptor* descriptor = new Descriptor();\\
			if(!(descriptor->get\_codec\_name()=="text"))\\
			continue;\\
			if (md\_stream->get\_descriptor(i, j, descriptor))\{
				payload\_size = descriptor->get\_payload\_size();\\
				if (md\_stream->is\_valid(descriptor->get\_flow\_id(), descriptor->get\_sequence\_number()))\{
					(*dc) += (descriptor->get\_payload());\\
					taken\_stream.resize(flows\_number*sequences\_number*(payload\_size+1));\\
					Uint8 current\_received\_data;\\
					Uint64 k;\\
					for (k=0; k<payload\_size; k++) \{\\
						dc->extract\_head(current\_received\_data);\\
						if (current\_received\_data != 0) \{\\
							Uint64 locate\_position = offset+i+(k*flows\_number);\\
							taken\_stream[locate\_position] = current\_received\_data;\\
							if (locate\_position > max\_dimension)\\
								max\_dimension = locate\_position;\\
						\}\\
					\}\\
					offset += flows\_number*k;\\
				\}\\
			\}\\
			else\{\\
				Uint64 k;\\
				for (k=0; k<payload\_size; k++) \{\\
					Uint64 locate\_position = offset+i+(k*flows\_number);\\
					taken\_stream[locate\_position] = ' ';\\
				\}\\
				offset += flows\_number*k;\\
			\}\\
		\}\\
	\}\\
	MemDataChunk* taken\_dc = new MemDataChunk();\\
	Uint8* temp\_container = new Uint8[max\_dimension+1];\\
	for (Uint64 i=0; i<max\_dimension+1; i++)\\
		temp\_container[i] = taken\_stream[i];\\
	taken\_dc->append\_data(max\_dimension+1, temp\_container);\\
	stream->set\_data(*taken\_dc);\\
\}\\
\end{code}

\begin{notabene}
La dicitura \textit{flow} viene usata con il significato di
`\emph{Descrizioni}'. Tale variazione rispetto alla nomenclatura presente in
letteratura è stata adottata al fine di evitare la possibile confusione dovuta alla somiglianza dei termini `\emph{Descrizioni}' e `\emph{Descrittori}'. In seguito, quindi, la parola `\emph{Flussi}' è da intendersi con il significato di `\emph{Descrizioni}'.
\end{notabene}

\section{Specifiche del codec testuale}
\label{sec:spec-text}
The MDC text codec function take a stream containing the text file and an empty
md\_stream that must be filled with flows and descriptors. The \textit{code()}
function cycles on flows and sequences numbers and, if the stream not be at end of file, create a new empty descriptor.
Now fills up the descriptor with corrects value as follows: 

\begin{center} \begin{tabular}{|c|c|}
\hline
Byte dimension & Content description\\
\hline \hline
3 bytes & ``MDC''\\
1 byte & version == 0\\
4 bytes & ``DESC''\\
32 bytes & stream\_id (hash md5 of the stream in version = 0)\\
1 byte & flow\_id\\
4 bytes & sequence\_number\\
1 byte & codec type == 1\\
4 bytes & codec parameters size\\
2 bytes & payload size\\
?? & payload\\
\hline
\end{tabular} \end{center}

First 8 byte are fixed and does not varying during the coding procedure.
Stream hash is an hash extracted from the stream's file name. Subsequently, there is the
real file name ending ``.mdc''. Flow id is an identifier of a single flow in a
MDC text stream. Is strongly reccomended to set this value to a base 2 multiple between 2 and 64 value. But is also possible to set this
value at any integer value from 1 to 64 flows. Coded file will contain the exact flows
number indicated. Sequence number is a value that takes count of single
descriptors, but it's used by algorithm as a user independent parameter. Codec
name is the name of codec: in this case it's ``text''. For text codec is not
necessary the ``codec parameter'' and has been added with size to 0 without any
real codec parameter. Payload size is the size of payload of a single descriptor and contains a value, for text implementation, between 25 and 55.000 bytes. This value represents the user preferred
payload size. The algorithm can vary it if it's necessary for load balacing between
descriptors and flows: the product between flows number and real descriptor payload must
approximate as best as possible text file size (it's a lower bound). The
payload is extracted from the stream from an initial point called ``offset'',
characters are extracted from a round robin algorithm to fill up a payload
ofapprossimatively size ``max\_payload\_size''. Then, payload is added to the
current descriptor. The ``offset'' parameter is used to monitorize this procedure. The stream contains a copy of text file contents.

\section{Specifiche del codec grafico}
The MDC image codec function take a stream containing the image file and an
empty md\_stream that must be filled up with flows and descriptors. The \textit{code()} function
cycles on flows and sequences numbers and, if the stream not is at end of file, create a new empty descriptor.
Now fills up the descriptor with corrects value as follows: 

\begin{center} \begin{tabular}{|c|c|}
\hline
Byte dimension & Content description\\
\hline \hline
3 bytes & ``MDC''\\
1 byte & version == 0\\
4 bytes & ``DESC''\\
32 bytes & stream\_id (hash md5 of the stream in version = 0)\\
1 byte & flow\_id\\
4 bytes & sequence\_number\\
1 byte & codec type == 2\\
4 bytes & codec parameters size\\
?? & codec parameters\\
2 bytes & payload size\\
?? & payload\\
\hline
\end{tabular} \end{center}

First 8 byte are fixed and does not varying during the coding procedure.
Stream hash is an hash extracted from the stream's file name. Subsequently, there is the
real file name ending ``.mdc''. Flow id is an identifier of a single flow in a
MDC text stream. Is strongly reccomended to set this value to a base 2 multiple between 2 and 64 value. But is also possible to set this
value at any integer value from 1 to 64 flows. Coded file will contain the exact flows
number indicated. Sequence number is a value that takes count of single
descriptors, but it's used by algorithm as a user independent parameter. Codec
name is the name of codec: in this case it's ``image''. For image codec the
``codec parameter'' and has been added with 5 bytes size. These bytes represents
the image width (2 bytes), height (2 bytes) and number of bits used for coding colours for each pixel (1 byte). Payload size is the size of payload of a
single descriptor and contains a value, for image implementation, between 10 and 18.330 pixels. This value represents the user preferred
payload size. The algorithm can vary it if it's necessary for load balacing between
descriptors and flows: the product between flows number and real descriptor payload must
approximate as best as possible image file size (it's a lower bound). The
payload is extracted from the stream from an initial point called ``offset'',
pixels are extracted from a round robin like algorithm to fill up a payload of
approssimatively size ``max\_payload\_size''. Then, payload is added to the current descriptor. The ``offset'' parameter is used to monitorize this procedure. The stream contains
a copy of image file contents.
