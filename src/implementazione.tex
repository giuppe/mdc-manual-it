\chapter{Implementazione}
\section{Codec testuale}
\label{cap:implementazione_codec}
Nel presente capitolo verrà descritto il funzionamento e l'implementazione
del codec testuale. L'analisi inizia da alcune considerazioni preliminari sul
file sorgente riguardanti la suddivisione delle informazioni secondo le
caratteristiche tipiche delle tecniche a descrizione multipla. Nella
implementazione di MDSP un file sorgente può essere suddiviso in uno o più
descrizioni, fino a un massimo di 64. Ogni descrizione, considerata
singolarmente, è in grado di descrivere l'intero file. \`E cioé sufficiente
alla corretta decodifica del file sorgente. Qualora il file sorgente venga
codificato in più di una descrizione sarà possibile la ricostruzione
perfetta del file esclusivamente nel caso in cui tutti le descrizioni giungano
a destinazione. Durante la procedura di decodifica del testo, MDSP versione 0.1
sostituisce un carattere \emph{spazio} qualora una lettera non giunga a
destinazione. Nel caso della decodifica di immagini viene sostituito il
pixel non giuntoa destinazione con uno apportunamente scelto e
poi interpolato. Oltre tali accorgimenti, è necessario prendere in
considerazione l'eventualità che giunga a destinazione una singola descrizione
con alcuni descrittori errati (caso peggiore). Tale eventualità si risolve
facendo sì che durante la codifica i descrittori, ma ancor più le descrizioni,
contengano dati provenienti 'da tutte le zone del file'. Al fine del
raggiungimento di tale scopo, vengono di seguito riportati gli indici necessari
e alcune semplici formule per poterli colcolare:
\begin{itemize}
 \item Dimensione di ogni descrizione: $$\frac{dimensione\_file}{numero\_descrizioni}$$
 \item Numero di descrittori per ogni descrizione: $$\frac{dimensione\_descrizione}{dimensione\_descrittore}$$
 \item Dimensione di ogni descrittore: $$\frac{dimensione\_descrizione}{numero\_descrittori}$$
\end{itemize}
Tali accorgimenti non sono da soli sufficienti per la realizzazione di un codec
pienamente compatibile MDC, ulteriori accorgimenti verranno illustrati con
maggiore dettaglio nel capitolo \ref{cap:codec_esempio}.
\\\\
La struttura delle classi C++ di gestione dei codec comprende classi astratte
che regolano lo scambio di dati tra i codec e il motore di streaming, nonché la
composizione delle strutture dati in memoria da utilizzare. Le classi a cui ci si riferisce sono contenute nella directory \textit{mdc/src/codecs/}. Di
seguito vengono elencate e brevemente descritte le classi astratte per la
gestione dei codec:
\begin{itemize}
 \item \textit{abstract\_codec\_parameter.h}: astrae i parametri del codec; per
 utilizzare tale classe è necessario specificare il comportamento di essa nelle
 funzioni \textit{serialize()} e \textit{deserialize()}.
 \item \textit{abstract\_md\_codec.h}: astrae il funzionamento del codec a
 descrizioni multiple; per utilizzare tale classe è necessario specificare il
 comportamento di essa nelle funzioni \textit{code()} e \textit{decode()}. In
 tali funzioni è contenuto il cuore del codec e pertanto necessitano un'analisi
 più dettagliata, si rimanda al capitolo \ref{cap:codec_esempio}.
 \item \textit{abstract\_stream.h}: astrae la gestione dei dati specifici per il
 tipo di file sorgente considerato; per utilizzare tale classe è necessario
 specificare il comportamento di essa nelle funzioni \textit{load\_from\_disk()}
 e \textit{save\_to\_disk()}.
\end{itemize}