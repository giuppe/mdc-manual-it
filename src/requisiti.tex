\chapter{Analisi dei Requisiti}

\section{Modello \emph{Layered}}

\section{Modello \emph{Multiple Descriptions}}

\section{Scopo}


\begin{itemize}
\item piattaforma per il test del protocollo mdc

\item progettazione e implementazione dei vari codec in un formato mdc-compatibile
\end{itemize}




\section{Codec}



\section{Definizione}


Un codec è un dispositivo (hardware o software) in grado di codificare e decodificare alcune informazioni in un formato di rappresentazione particolare. In genere il termine è usato in relazione a flussi multimediali (audio/video) processati tramite un elaboratore elettronico.






\section{Tipologie di codifica}


Le codifiche possono essere catalogate in differenti tipologie, ad esempio la tipologia 'Layered' ('a strati'), comune a tutti i codec MPEG.



Il Multiple Description Coding prevede che:
\begin{enumerate}
\item il flusso informativo sia suddiviso in più frammenti, detti 'Descrittori', con circa la stessa quantità di informazioni
\item i descrittori costituiscano due o più sotto-flussi;
\item i descrittori di un solo sotto-flusso, qualunque sia, devono bastare a ricostruire l'informazione, per lo meno con una bassa qualità
\item se vengono ricevuti due o più sotto-flussi, la qualità dell'informazione decodificata aumenta 
\end{enumerate}





\section{Differenze tra MDC e Layered}


La principale differenza tra la tipologia di codifica 'a descrittori multipli' e quella 'a strati' sta nella differente organizzazione delle informazioni: i frammenti ottenuti con una codifica MDC hanno potenzialmente lo stesso contenuto informativo in termini di quantità, mentre nei codec 'layered' esiste un frammento 'base', più importante degli altri, e diversi frammenti più piccoli di 'avanzamento'.






\section{Testbed}



\section{Organizzazione}


Il testbed è composto da un insieme di classi, suddivise in directory:



\begin{itemize}
\item 'common': classi che facilitano la realizzazione delle proprie applicazioni; comprendono utility per accedere alla rete, per simulare un comportamento multi-tasking, e altro.

\item 'codecs': contiene sia classi astratte, da usare come base per i codec reali, sia classi di utilità, come il registro dei codec, utilizzato per auto-caricare il codec adatto a seconda del flusso informativo

\item 'messages': l'iniseme dei messagi di base
\end{itemize}




\section{Implementazione}


Il testbed è stato sviluppato in C/C++, anche se, seguendo le specifiche, è possibile implementare una qualunque parte del sistema con qualsiasi linguaggio.






\section{Messaggi}


Le componenti del sistema che si trovano su elaboratori differenti comunicano tra loro per meggo di 'messaggi'. Tali messaggi rappresentano una particolare richiesta di servizi, oppure una corrispondente risposta.

Un messaggio possiede una 'tipologia' e un certo numero di 'argomenti', dipendenti dal tipo di messaggio.

Il tipico messaggio è composto da un header di 8 byte, cosi' suddivisi:



\begin{itemize}
\item 3 byte per la stringa MDC

\item 1 byte per la versione del protocollo

\item 4 byte per il tipo di messaggio
\end{itemize}

I messaggi possono avere un singolo parametro, oppure molti. Per 'parametro' qui ci si riferisce al parametro del messaggio, ovvero una stringa null-terminated, che a sua volta può contenere uno o più parametri a livello applicativo.






\subsection{LIST}
%

LIST



Composizione:



\begin{itemize}
\item header MDC

\item null-terminated string
\end{itemize}

la stringa può contenere una espressione regolare per la definizione del nome del flusso ricercato






\subsection{ALST}
%

Answer LiST



Lista di parametri






\subsection{SREQ}
%

Stream REQuest



Singolo parametro



\begin{itemize}
\item hash dello stream
\end{itemize}




\subsection{ASRQ}
%

Answer Stream ReQuest



Singolo parametro






\subsection{SINF}
%

Stream INFormation



Singolo parametro






\subsection{ASNF}
%

Answer Stream iNFormation



Singolo parametro






\subsection{PARM}
%

PARaMeters






\subsection{KALV}
%

Keep ALiVe



Singolo parametro






\subsection{PEER}
%

PEER



Singolo parametro






\subsection{APER}
%

Answer PEeR



Lista di parametri






\section{Descrittore}


Un 'Descriptor' è, nell'ambito di questo sistema, una unità di dati di un flusso multimediale.

Il flusso multimediale è suddiviso in sotto-flussi detti 'descrizioni', secondo le specifiche della codifica md; ogni descrizione è composta da un certo numero di descrittori, con la stessa relazione che c'è tra un filmato e un fotogramma.

Il Descriptor contiene varie informazioni, come il flusso originario, il numero di descrizione, e la sequenza temporale. Il payload non è definito e pertanto è permessa una maggiore libertà in fase di definizione dei codec.






\section{Casi d'uso}


Chiameremo Alice un generico nodo della rete che vuole effettuare le azioni, e Bob il peer a cui queste comunicazioni sono dirette.

Ci sono poi anche altri amici, come Carlina, David ed Emy, che condividono le stesse passioni di Alice e Bob.


\subsection{Richiesta di un flusso multimediale}
%

Alice ha bisogno di un flusso multimediale, e sa già che Bob ce l'ha, tutto o in parte (lo sa perché ha già visto la lista dei flussi di Bob).



Quindi Alice invia un messaggio SREQ, specificando come parametro l'hash univoco del flusso, il numero della descrizione, e una sequenza di descrittori.



Bob quindi invierà un messaggio ASRQ, specificando come parametro il timeout di keep-alive; Bob, infatti, continuerà ad inviare il flusso di dati finché riceverà pacchetti di KALV (keep-alive) da Alice, supponendola disconnessa allo scadere del timeout. 






\subsection{Richiesta della lista dei flussi}
%

\begin{itemize}
\item Alice ha bisogno della lista dei flussi disponibili da Bob. Invia un messaggio LIST, non parametro vuoto, e riceve da Bob un messaggio ALST (Answer LiST) contenente un vettore con tutti i nomi e gli hash dei flussi che egli possiede;

\item Alice sta cercando un particolare flusso, di cui sa parte del nome ('pippo'); invia un messaggio LIST a Bob con parametro '*pippo*', e Bob risponderà con un ALST contenente informazioni su tutti i file che contengono la parola 'pippo' nel nome.
\end{itemize}




\subsection{Richiesta di informazioni su un flusso}
%

Alice sa che un determinato flusso (di cui possiede l'hash) è posseduto da Bob; invia quindi un messaggio SINF (Stream INFormation), per avere informazioni specifiche del flusso, ad esempio la durata, la qualità di codifica, ecc. Bob invia un messaggio ASNF contenente la risposta.






\subsection{Invio di parametri di rete}
%

Alice deve informare Bob di cambiamenti che stanno avvenendo dalla sua parte; per esempio, sta cambiando rete e non vuole interrompere lo streaming; oppure c'è stato un problema con il router ed è necessario abbassare la velocità di invio del flusso. In tutti questi casi, Alice invia un messaggio PARM a Bob, specificando i nuovi parametri.






\subsection{Richiesta di altri peer con lo stesso flusso}
%

Alice sta scaricando un flusso da Bob, ma avverte la necessità di frequentare altre persone, che magari condividono lo stesso flusso; invia quindi un messaggio PEER a Bob, seguito dall'hash univoco del flusso. Bob raccoglie le idee su tutti gli amici che conosce, o che ha conosciuto tramite lo scaricamento del flusso in oggetto, per esempio Carlina, da cui Bob ha scaricato originariamente il flusso, David, che ha scaricato il flusso tempo fa, ed Emy che lo sta scaricando ora. Bob quindi prepara un messaggio APER contenente l'indirizzo ip e la porta di Carlina, David ed Emy e lo invia ad Alice.






\section{MDSP}


MDSP sta per Multiple Description Stream Protocol, ed è un protocollo per lo scambio di stream di vario tipo (contenuti multimediali) organizzati secondo una meta-codifica a descrittori multipli.





