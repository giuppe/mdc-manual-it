

\chapter{Introduzione}

\section*{Multiple Description Coding}
Il \emph{Multiple Description Coding} è una tecnica di meta-codifica che prevede
la suddivisione di un flusso di dati generico (che può rappresentare testo,
immagini, audio o video) in più descrizioni (\emph{Descriptions}); ogni descrizione
è costruita in modo tale da permettere di ricostruire il flusso originale,
sebbene con qualità inferiore.

Una tecnica di meta-codifica differente e molto diffusa è quella \emph{layered}
(``a strati''), che sta alla base di codec come mpeg. Questa tecnica prevede la
suddivisione di un flusso di dati in uno ``strato base'' e molteplici ``strati
di avanzamento'', che contengono minori informazioni. \`E necessario che lo
``strato base'' abbia priorità massima, mentre la consegna degli altri strati
può avvenire con bassa priorità. Questo tipo di approccio funziona molto bene
quando la rete può utilizzare differenti ``priorità'' per il trasporto degli
strati: per questo i codec \emph{layered} sono molto diffusi su reti
``intelligenti'' (come DVB-*) e meno su Internet, dove il trasporto dei
pacchetti avviene in modo \emph{best effort}.



%Multiresolution or layered system is good when network provides a way to treat
%differently some packet.

%MD coding can be applied to ad-hoc networks.

%Ad-hoc networks: large number of unreliable devices, packets can follow
%multiple paths but the single path can fail: md coding supports this scenario.

