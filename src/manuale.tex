\chapter{Manuale}
Nel presente capitolo verranno illustrati i principali comandi per la
transcodifica e la gestione del protocollo MDSP.

\section{Comandi principali}
I principali comandi riguardanti l'applicazione MDSP versione 0.1 sono relativi
alle operazioni di codifica, decodifica e avvio in modalità server e client. I
comandi relativi alla codifica e alla decodifica di seguito riportati sono
relativi all'uso di un file testuale. L'esempio è comunque valido nel caso si
vogliano utilizzare file contenenti immagini.

\begin{itemize}
  \item Codifica: \begin{code} mdc --input text\_file.txt --code --codec text
  --output text\_file.mdc \end{code}
  \item Decodifica: \begin{code} mdc --input text\_file.mdc --decode --codec
  text --output new\_text\_file.txt \end{code}
  \item Modalità server: \begin{code} mdc --daemon \end{code}
  \item Modalità client: \begin{code} mdc \end{code}
\end{itemize}


