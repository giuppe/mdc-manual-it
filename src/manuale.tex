\chapter{Manuale utente}
\section{Documentazione codec}
Il protocollo di streaming a descrittori multipli fa nascere la necessit\`a di progettare e realizzare codec \emph{codificatori/decodificatori} fatti su misura. Nel presente capitolo verranno introdotti gli accorgimenti necessari alla realizzazione di codec compatibili con le modalit\`a operative di MDSP. Viene rimandata ad un altro capitolo la presentazione, a titolo esplicativo, del codec pi\`u semplice da implementare ed utilizzare: il \emph{codec testuale}.

\subsection{Caratteristiche necessarie per la realizzazione di un codec}
Affinch\`e un codec possa funzionare con il protocollo MDC (\emph{Multiple Description Coding}) è necessario che possieda delle caratteristiche ben precise. La codifica MDC ha come obiettivo finale la descrizione di un file in modo tale che, in caso di perdita parziale dei dati che lo costituiscono, il file stesso sia ugualmente interpretabile. L'assoluta interpretabilit\`a del file costituisce un vincolo molto stringente e sono necessari codec con particolari accorgimenti. La versione 0.1 di MDSP gestisce testi semplici e immagini, e consente di visualizzare il contenuto di un file anche se si verificano errori nel trasferimento dello stesso attraverso una rete di telecomunicazione. Durante la fase di codifica il file sorgente viene suddiviso in descrizioni (chiamate anche flussi) e descrittori:
\begin{itemize}
 \item Un descrittore \`e la pi\`u piccola unit\`a di misura del contenuto di un file. L'insieme di pi\`u descrittori costituisce una descrizione.
 \item Una descrizione pu\`o contenere parte o tutte le informazioni di un file (con ``informazioni'' si intendono i dati costituenti il file).
\end{itemize}
Il file sorgente pu\`o essere suddiviso in uno o pi\`u descrizioni, fino a un massimo di 64. Ogni descrizione, presa singolarmente, \`e in grado di descrivere l'intero file. \`E cio\`e sufficiente alla corretta decodifica del file. La codifica di un file in una singola descrizione fa s\`i che la sua corretta decodifica restituisca un file clone del file sorgente. Se, invece, il file sorgente viene codificato in pi\`u di una descrizione, allora sarà possibile la ricostruzione perfetta del file esclusivamente nel caso in cui tutti le descrizioni giungano a destinazione.
\\
Le descrizioni costituiscono differenti codifiche e devono essere il più possibile ortogonali fra loro, in modo da poter essere indipendenti l'una dall'altra,pur riferendosi tutte allo stesso file sorgente.
\\
Nel caso in cui qualche descrizione venga persa, sia corrotta, o non venga inviata dalla sorgente, sarà comunque possibile decodificare il file ed otterene un output sufficientemente corretto, tale da poter essere intepretato. \`E sufficiente la ricezione di una sola descrizione per la corretta ricostruzione del file sorgente. La ricezione di due o pi\`u descrizioni migliora l'accuratezza del file decodificato. Il file sorgente viene ``\emph{descritto}'' con pi\`u precisione.  Tale risultato viene raggiunto grazie alla creazione di codec che tengono in considerazione il fattore umano nell'interpretazione di vari tipi di informazioni:
\begin{itemize}
 \item Nel caso del testo, una parola con una lettera mancante viene facilmente interpretata dal lettore. Il cervello umano sostituisce la lettera mancante con quella corretta basandosi sul senso della parola stessa o della frase.
 \item Nel caso delle immagini, la mancanza di pixel non rende l'immagine inutilizzabile. All'occhio umano saranno presenti delle lievi alterazioni, ma, anche questa volta, il contesto rende ugualmente interpretabile l'immagine e le informazioni che essa contiene.
\end{itemize}
Il funzionamento fin qu\`i illustrato della suddivisione del file in descrizioni si pu\`o applicare allo stesso modo per descrivere la suddivisione di ogni descrizione in descrittori. Come precentemente accennato, i descrittori sono la pi\`u piccola unit\`a di memorizzazione dei dati costituenti il file sorgente. Tale affermazione significa che, se viene perso un descrittore durante il trasferimento, oppure i suoi dati non sono corretti, esso può essere semplicemente non considerato. Ci\`o permette un'estrema flessibilit\`a di decodifica. Sostanzialmente, il decodificatore deve adattarsi alla tipologia e quantit\`a di dati collezionati e creare un file di output. Tale file sar\`a perfetto se non si sono verificate perdite, o ugualmente interpretabile in presenza di perdite.
\\
MDSP versione 0.1 sostituisce un carattere \emph{spazio} se una lettera non giunge a destinazione (per i testi), oppure un pixel apportunamente scelto e poi interpolato per ogni pixel non giunto a destinazione (per le immagini). Oltre tali accorgimenti, è necessario prendere in considerazione l'eventualità che giunga a destinazione una singola descrizione con alcuni descrittori errati (caso peggiore). Tale eventualit\`a si risolve facendo s\`i che durante la codifica i descrittori, ma ancor pi\`u le descrizioni, contengano dati provenienti \emph{da tutte le zone del file}. Al fine del raggiungimento di tale scopo, vengono di seguito riportati gli indici necessari e semplici formule per poterli colcolare:
\begin{itemize}
 \item Dimensione di ogni descrizione: $$\frac{dimensione\_file}{numero\_descrizioni}$$
 \item Numero di descrittori per ogni descrizione: $$\frac{dimensione\_descrizione}{dimensione\_descrittore}$$
 \item Dimensione di ogni descrittore: $$\frac{dimensione\_descrizione}{numero\_descrittori}$$
\end{itemize}
Tali accorgimenti non sono da soli sufficienti per realizzare un codec pienamente compatibile con le specifiche MDC, altri accorgimenti verranno illustrati con maggiore dettaglio con un esempio successivamente.