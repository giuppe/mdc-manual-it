\chapter{Manuale}
Nel presente capitolo verranno illustrati i principali comandi per la
transcodifica e la gestione dell'applicazione.

\section{Comandi principali}
I principali comandi riguardanti l'applicazione MDSP versione 0.1 sono relativi
alle operazioni di codifica, decodifica e avvio in modalità server e client. I
comandi relativi alla codifica e alla decodifica di seguito riportati sono
relativi all'uso di un file testuale. L'esempio è comunque valido nel caso si
vogliano utilizzare file contenenti immagini.

\begin{itemize}
  \item Codifica: \texttt{mdc --input text\_file.txt --code --codec text
  --output text\_file.mdc}\\ Per la codifica è strettamente necessario
  specificare il file sorgente (\emph{input}), il file di output, l'operazione desiderata (\emph{code}) e il
tipo di codec da utilizzare (\emph{text}). Nel paragrafo \ref{sec:opzioni} vengono riportate tutte le opzioni disponibili. 
  \item Decodifica: \texttt{mdc --input text\_file.mdc --decode --codec
  text\\ --output new\_text\_file.txt}\\ Per la decodifica è strettamente
  necessario specificare il file sorgente (\emph{input}), il file di output, l'operazione desiderata (\emph{decode}) e il
tipo di codec da utilizzare (\emph{text}). Nel paragrafo \ref{sec:opzioni} vengono riportate tutte le opzioni disponibili.
  \item Modalità server: \texttt{mdc --daemon}\\ Lanciando da linea di comando
  MDSP versione 0.1 in modalità server, l'applicazione viene configurata come `erogatore di file' e si pone in ascolto
sulle porte UDP 5551 per i segnali di controllo e 5552 per lo scambio dai dati. Nel capitolo \ref{cap:protocollo} viene illustrato nel dettaglio il
funzionamento e la configurazione della modalità di esecuzione server.
  \item Modalità client: \texttt{mdc --connect}\\ Il parametro
  \texttt{--connect} è opzionale e specifica l'indirizzo IPv4 del server al
  quale si intende collegarsi. In assenza del parametro MDSP versione 0.1
  caricherà l'IP di default specificato all'interno del file di configurazione
  \texttt{config.xml} (vedasi paragrafo \ref{cap:xml}).
\end{itemize}

\section{Opzioni}
\label{sec:opzioni}
Di seguito vengono elencate e descritte tutte le opzioni dell'applicazione:

\begin{itemize}
  \item \texttt{--input}: (\emph{parametro obbligatorio}) è il
  nome del file sorgente, il nome comprende anche l'eventuale path completo;
  \item \texttt{--output}: (\emph{parametro obbligatorio}) è il
  nome del file di destinazione (il file decodificato), il nome comprende anche l'eventuale path
  completo.
  \item \texttt{--code}: (\emph{parametro obbligatorio}) indica
  l'operazione da effettuare, in questo caso codifica del file sorgente nel suo equivalente
  \emph{`.mdc'}.
  \item \texttt{--input}: (\emph{parametro obbligatorio}) indica
  l'operazione da effettuare, in questo caso decodifica del file \emph{`.mdc'}
  nel formato del file di output.
  \item \texttt{--codec}: (\emph{parametro obbligatorio}) specifica il tipo di
  codec da utilizzare. MDSP versione 0.1 può utilizzare un codec testuale (\emph{text})
  e un codec grafico (\emph{image}).
  \item \texttt{--flows}: specifica il numero di descrizioni in cui verrà
  codificato il file corgente. Tale opzione è utilizzabile esclusivamente insieme all'opzione
  \texttt{--code}, altrimenti verrà ignorata. \`E possibile specificare un
  numero intero di descrizioni compreso tra 1 e 64, se tale opzione non viene
  specificata MDSP procederà alla codifica del file sorgente utilizzando 2
  descrizioni.
  \item \texttt{--payload}: specifica una preferenza sulla quantità di dati che
  si desidera vengano trasportati in ogni descrittore. MDSP può modificare tale
  valore allorquando l'utente specifichi un valore non compatibile con le
  specifiche. Tale opzione è utilizzabile esclusivamente insieme all'opzione
  \texttt{--code}, altrimenti verrà ignorata. Se si utilizza il codec testuale
  tale valore può essere impostato in un intervallo compreso tra 25 e 55000
  (l'unità di misura è il carattere). In caso venga utilizzato il codec grafico
  l'intervallo ammesso è compreso tra 10 e 18330 (l'unità di misura è il
  pixel). Se tale opzione non viene specificata MDSP procederà alla codifica
  del file sorgente impostando il parametro al valore 1000 indipendentemente dal codec utilizzato.
  \item \texttt{--daemon}: avvia MDSP in modalità server.
  \item \texttt{--connect}: specifica l'indirizzo IP del ``server'' al quale MDSP
  dovrà collegarsi quando opera in modalità ``client''.
  \item \texttt{--help}: visualizza una breve descrizione delle modalità
  operative e dei parametri ammessi.\\
\end{itemize}

\begin{notabene}
Lanciando MDSP senza la specifica di alcun parametro viene
avviata l'applicazione in modalità ``client'' attendendo una connessione con un peer
il cui inirizzo IP può essere fornito da linea di comando. In assenza del
parametro \texttt{--connect} viene caricato per dafeult l'IP 192.168.0.30
(specificato nel file di configurazione \texttt{config.xml}, vedere paragrafo \ref{cap:xml}).
\end{notabene}

\section{Configurazione}
\label{cap:xml}
Di seguito viene brevemente descritto il file di configurazione XML che
memorizza i parametri principali necessari al funzionamento di MDSP. Tale file
è contenuto nella directory principale dell'applicazione con nome
\texttt{config.xml}. A titolo di esempio viene descritto il seguente file:

\begin{code}
<network>\\
    <control\_port>5551</control\_port>\\
    <data\_port>5552</data\_port>\\
</network>\\
<repository>\\
    <path>./shared</path>\\
</repository>\\
<server>\\
    <address>192.168.0.30</address>\\
</server>\\
\end{code}

La sezione \texttt{<network>} contiene il numero delle porte di controllo e
dati che vengono utilizzate dal sistema. La sezione \texttt{<repository>}
contiene la directory nella quale verrà salvato il file codificato. Infine la
sezione \texttt{<server>} contiene l'indirizzo IPv4 necessario alla eventuale
modalità di funzionamento ``client'' dell'applicazione. L'ultima sezione
non è obbligatoria. \`E infatti possibile fornire un indirizzo IP anche da
linea di comando tramite l'opzione di comando \texttt{--connect}, nel caso in
cui tale opzione non venga specificata verrà caricato il valore di default
contenuto nel file di configurazione. Nel caso in cui anche il file di
configurazione non contenga alcun indirizzo IP viene segnalato all'utente
l'errore provocatosi e viene automaticamente impostato il valore 192.168.0.30
nel file di configurazione. Sarà quindi necessario un riavvio dell'applicazione.
\\\\
Esempi di utilizzo:

\begin{code}
mdc --input input\_file.bmp --codec image --code --flows 4 --payload
2000\\ --output output\_file.mdc\\\\
mdc --connect 192.168.0.124\\\\
mdc --input input\_file.mdc --codec image --decode --output
new\_output\_file.bmp
\end{code}