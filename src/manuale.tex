\chapter{Manuale}
Nel presente capitolo verranno illustrati i principali comandi per la
transcodifica e la gestione del protocollo MDSP.

\section{Comandi principali}
I principali comandi riguardanti l'applicazione MDSP versione 0.1 sono relativi
alle operazioni di codifica, decodifica e avvio in modalità server e client. I
comandi relativi alla codifica e alla decodifica di seguito riportati sono
relativi all'uso di un file testuale. L'esempio è comunque valido nel caso si
vogliano utilizzare file contenenti immagini.

\begin{itemize}
  \item Codifica: \texttt{mdc --input text\_file.txt --code --codec text
  --output text\_file.mdc}\\ Per la codifica è strettamente necessario
  specificare il file sorgente (\emph{input}), il file di output, l'operazione desiderata (\emph{code}) e il
tipo di codec da utilizzare (\emph{text}). Nel paragrafo \ref{sec:opzioni} vengono riportate tutte le opzioni disponibili. 
  \item Decodifica: \texttt{mdc --input text\_file.mdc --decode --codec
  text\\ --output new\_text\_file.txt}\\ Per la decodifica è strettamente
  necessario specificare il file sorgente (\emph{input}), il file di output, l'operazione desiderata (\emph{decode}) e il
tipo di codec da utilizzare (\emph{text}). Nel paragrafo \ref{sec:opzioni} vengono riportate tutte le opzioni disponibili.
  \item Modalità server: \texttt{mdc --daemon}\\ Lanciando da linea di comando
  MDSP versione 0.1 in modalità server, l'applicazione viene configurata come `erogatore di file' e si pone in ascolto
sulle porte UDP 5551 per i segnali di controllo e 5552 per lo scambio dai dati.
  \item Modalità client: \texttt{mdc}
\end{itemize}

Nel capitolo \ref{cap:protocollo} viene illustrato nel dettaglio il
funzionamento e la configurazione della modalità di esecuzione `server'.

\section{Opzioni}
\label{sec:opzioni}
Di seguito vengono elencate e descritte tutte le opzioni dell'applicazione:

\begin{itemize}
  \item \texttt{--input}: (\emph{parametro obbligatorio}) è il
  nome del file sorgente, il nome comprende anche l'eventuale path completo;
  \item \texttt{--output}: (\emph{parametro obbligatorio}) è il
  nome del file di destinazione (il file decodificato), il nome comprende anche l'eventuale path
  completo.
  \item \texttt{--code}: (\emph{parametro obbligatorio}) indica
  l'operazione da effettuare, in questo caso codifica del file sorgente nel suo equivalente
  '.mdc'.
  \item \texttt{--input}: (\emph{parametro obbligatorio}) indica
  l'operazione da effettuare, in questo caso decodifica del file '.mdc' nel formato del file di
  output.
  \item \texttt{--codec}: (\emph{parametro obbligatorio}) specifica il tipo di
  codec da utilizzare. MDSP versione 0.1 può utilizzare un codec testuale (\emph{text})
  e un codec grafico (\emph{image}).
  \item \texttt{--flows}: specifica il numero di descrizioni in cui verrà
  codificato il file corgente. Tale opzione è utilizzabile esclusivamente insieme all'opzione
  \emph{--code}, altrimenti verrà ignorata. \`E possibile specificare un numero
  intero di descrizioni compreso tra 1 e 64, se tale opzione non viene
  specificata MDSP procederà alla codifica del file sorgente utilizzando 2
  descrizioni.
  \item \texttt{--payload}: specifica una preferenza sulla quantità di dati che
  si desidera vengano trasportati in ogni descrittore. MDSP può modificare tale
  valore allorquando l'utente specifichi un valore non compatibile con le
  specifiche. Tale opzione è utilizzabile esclusivamente insieme all'opzione
  \emph{--code}, altrimenti verrà ignorata. Se si utilizza il codec testuale
  tale valore può essere impostato in un intervallo compreso tra 25 e 55000
  (l'unità di misura è il carattere). In caso venga utilizzato il codec grafico
  l'intervallo ammesso è compreso tra 10 e 18330 (l'unità di misura è il
  pixel). Se tale opzione non viene
  specificata MDSP procederà alla codifica del file sorgente impostando il
  parametro al valore 1000 indipendentemente dal codec utilizzato.
  \item \texttt{--daemon}: avvia MDSP in modalità server.
  \item \texttt{--help}: visualizza una breve descrizione delle modalità
  operative e dei parametri ammessi.
\end{itemize}

Esempi di utilizzo:

\begin{code}
mdc --input input\_file.bmp --codec image --code --flows 4 --payload
2000\\ --output output\_file.mdc\\\\
mdc --input input\_file.mdc --codec image --decode --output
new\_output\_file.bmp
\end{code}