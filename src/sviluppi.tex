\chapter{Sviluppi Futuri}
\label{cap:sviluppi_futuri}
Esistono alcune idee di sviluppo che non sono state implementate nella versione
corrente, ma potrebbero avere sviluppi futuri interessanti.


\section{Web cooperativo}
In fase di progettazione è stata presa in considerazione la possibilità di un
eventuale utilizzo delle tecniche di codifica/decodifica MD compatibili
applicate ai contenuti web. Partendo dai pressupposti di funzionamento di MDSP
è possibile lo sviluppo futuro di un'applicazione il cui obiettivo sia quello
di effettuare il download di un'intera pagina web i cui contenuti siano file
testuali, grafici, o qual si voglia, tutti codificati secondo le specifiche MDC.
Tali contenuti web si potrebbero a quel punto scambiare tra più peer
permettendo così di realizzare una particolare forma di \emph{navigazione web
cooperativa}. Anche se in via del tutto prematura è auspicabile un sistema di ritrasmissione di descrittori eventualmente persi per quei file dal
'contenuto sensibile' quali, ad esempio, file '\emph{.html}'o '\emph{.php}'. In
quest'ultima considerazione si deve tenere conto della necessità di assoluta
assenza di errori per file di questo tipo, in quanto la presenza di un errore
può provocare il malfunzionamento di una \emph{marcatura} o di uno script fino
all'eventuale corruzione totale del file. Va inoltre considerata l'eventualità
della creazione di un sistema di sicurezza che non permetta lo scambio di
descrittori relativi a file personali dell'utente: non è ammissibile il
download cooperativo di file relativi, ad esempio, alla posta elettronica. Al
limite è ammissibile uno scenario cooperativo in cui i descrittori presi in
considerazione vengano crittografati e poi scambiati.

\section{Codec video}
\label{cap:codec_video}
Durante la realizzazione del codec testuale e del codec grafico, si è presa in
considerazione la possibile realizzazione futura di un codec video.

\subsection{Problematiche}

In seguito agli studi fatti è emersa la necessità che il codec video possieda
alcune peculiarità differenti dalla attuale tecnologia ed approccio di codifica, in
particolare, ci si riferisce alle seguenti considerazioni:

\begin{enumerate}
\item Lo stream video codificato deve conterene informazioni di tipo
esclusivamente 'visuale'. Non deve, cioé, contenere informazioni audio e video
contemporaneamente.

\item La codifica video basata su GOP (Group of Pictures) o GOF (Group of
Frames) è affetta da probabile tramissione di errori, i quali deteriorano il
file di output nell'istante in cui vengono ricevuti e per un certo lasso di
tempo. Tale problema è evidente nella codifica video 'layered'. Un codec video
MDC compatibile deve invece manifestare una perdita di qualità visiva
esclusivamente nell'istante in cui vegono perse informazioni.

\item La codifica layered prevede la presenza di un flusso principale e di
alcuni sottoflussi sparsi all'interno di un singolo GOP o sparsi tra più GOP.
La realizzazione di un codec video MDC compatibile, invece, sovverte tale
approccio codificando il file sorgente in uno o più descrizioni tutte egualmente
(e separatamente) decodificabili.
\end{enumerate}

\subsection{Soluzioni}

Le problematiche appena descritte nascono da analisi effettuate relative alla
possibile applicazione della codifica \emph{MDC based} su file video già
codificati con algoritmi 'layered'. Le possibili soluzioni sono le seguenti:

\begin{itemize}
  \item Realizzazione di un codec \emph{MDC puro}: ossia un codec per la cui
  realizzazione è richiesta la conversione del file codificato in un file RAW e
  la sua successiva codifica secondo gli standard MD.
  \item Realizzazione di un codec \emph{ibrido}: ossia un codec che riesca a
  'spacchettare' la codifica layered, estragga i dati utili e li utilizzi per
  creare descrizioni virtualmente indipendenti.
\end{itemize}

La prima di tali soluzioni rispetta pienamente le specifiche MDC in quanto
permette di manipolare più facilmente i dati del file sorgente per creare
descrizioni multiple. Non è però facilmente applicabile a causa della
disponibilità estremamente esigua di documentazione in letteratura a riguardo. Richiede
altresì la realizzazione di \emph{regole di trasformazione} tra i dati RAW e le
descrizioni corrispondenti, si dovrebbe realizzare inoltre un sistema di
compressione di tali dati.

La soluzione basata su codec \emph{ibrido} comporta una notevole complessità
relativa alla fase di decodifica del video: spacchettizzazione dei GOP,
esclusione dell'audio e compensazione degli errori. Sarebbe necessaria una
ulteriore analisi sulla previsione interframe e interGOP utilizzata dai codec
layered: è necessario annullare tali informazioni poiché ogni descrizione
contiene solo alcune porzioni dei dati su cui tali statistiche sono state calcolate.
Sarebbe inevitabile, inoltre, la creazione di ulteriore ridondanza relativa
alla presenza del layer base in tutte le descrizioni MDC, oltre le informazioni
di arricchimento. \`E però indiscutibile l'eventuale enorme diffusione che avrebbe
tale codec in quanto permetterebbe l'immediata impiegabilità con la quasi totalità dei file video oggi esistenti.

\section{Codec audio}
Anche le informazioni audio vengono oggi codificate seguendo il modello
layered, così come accade per i contenuti video. Anche in questo caso, quindi,
è necessario far proprie tecniche di decodifica la cui applicazione comporta
alcuni problemi teorico/pratici già descritti in appendice
\ref{cap:codec_video}.

\section{Rete distribuita}
I presupposti di funzionamento del protocollo MDSP sono stati
scelti nell'ottica di un eventuale utilizzo anche in reti distribuite con presenza di
errori e/o topologie variabili.

\subsection{Ad-Hoc Networks}
Le reti Ad-Hoc sono caratterizzate dalla mancanza di una \emph{autorità}
centrale, un mezzo condiviso con una banda potenzialmente bassa e mobilità
dei nodi. Queste problematiche rendono difficile realizzare connessioni con
qualità sufficiente per applicazioni real-time.
La presenza di connessioni in grado di utilizzare molteplici percorsi possibili
tra la sorgente e la destinazione può aumentare le performance: il protocollo
MDSP supporta questo approccio, permettendo richieste di ``frammenti'' di
flussi multimediali a più nodi contemporaneamente.

\section{Regolazione della banda}
\`E inoltre possibile realizzare un modulo di regolazione di banda. Durante il
trasferimento di un file può essere necessario modificare la banda utilizzata. \`E pensabile la progettazione di un modulo a supporto
dell'applicazione che monitorizzi la banda disponibile, il quantitativo di file
richiesti o la qualità di ricezione e adegui la richiesta delle descrizioni in
modo tale da rispettare i vincoli presi in esame.

\section{Applicazioni destinate all'utente non esperto}
Infine, ma per questo non meno importante, si rimanda a sviluppi futuri la
creazione di una interfaccia grafica che guidi in modo amichevole l'utente nei
passi di configurazione dell'applicazione, dell'esecuzione della stessa e
durante le fasi di codifica/decodifica. In futuro sarebbe possibile la
creazione di un'applicazione esclusivamente dedicata allo scambio di file
\emph{``.mdc''}, e quindi completamente general purpose, al pari di altri
sistemi peer-to-peer già esistenti.