\chapter{Sviluppi Futuri}

Esistono alcune idee di sviluppo che non sono state implementate nella versione
corrente, ma potrebbero avere sviluppi futuri interessanti.


\section{Web cooperativo}

\section{Codec video}
Durante la realizzazione del codec testuale e del codec grafico, si è presa in
considerazione la possibile realizzazione futura di un codec video.

\subsection{Problematiche}

In seguito agli studi fatti è emersa la necessità che il codec video possieda
alcune peculiarità differenti dalla attuale tecnologia ed approccio di codifica, in
particolare, ci si riferisce alle seguenti considerazioni:

\begin{enumerate}
\item Lo stream video codificato deve conterene informazioni di tipo
esclusivamente 'visuale'. Non deve, cioé, contenere informazioni audio e video
contemporaneamente.

\item La codifica video basata su GOP (Group of Pictures) o GOF (Group of
Frames) è affetta da probabile tramissione di errori, i quali deteriorano il
file di output nell'istante in cui vengono ricevuti e per un certo lasso di
tempo. Tale problema è evidente nella codifica video 'layered'. Un codec video
MDC compatibile deve invece manifestare una perdita di qualità visiva
esclusivamente nell'istante in cui vegono perse informazioni.

\item La codifica layered prevede la presenza di un flusso principale e di
alcuni sottoflussi sparsi all'interno di un singolo GOP o sparsi tra più GOP.
La realizzazione di un codec video MDC compatibile, invece, sovverte tale
approccio codificando il file sorgente in uno o più descrizioni tutte egualmente
(e separatamente) decodificabili.
\end{enumerate}

\subsection{Soluzioni}

Le problematiche appena descritte nascono da analisi effettuate relative alla
possibile applicazione della codifica \emph{MDC based} su file video già
codificati con algoritmi 'layered'. Le possibili soluzioni sono le seguenti:

\begin{itemize}
  \item Realizzazione di un codec \emph{MDC puro}: ossia un codec per la cui
  realizzazione è richiesta la conversione del file codificato in un file RAW e
  la sua successiva codifica secondo gli standard MDC.
  \item Realizzazione di un codec \emph{ibrido}: ossia un codec che riesca a
  'spacchettare' la codifica layered, estragga i dati utili e li utilizzi per
  creare descrizioni virtualmente indipendenti.
\end{itemize}

La prima di tali soluzioni rispetta pienamente le spacifiche MDC in quanto
permette di manipolare più facilmente i dati del file sorgente per creare
descrizioni multiple. Non è però facilmente applicabile a causa della
disponibilità estremamente esigua di documentazione in letteratura a riguardo. Richiede
altresì la realizzazione di \emph{regole di trasformazione} tra i dati RAW e le
descrizioni corrispondenti, si dovrebbe realizzare inoltre un sistema di
compressione di tali dati.
\\\\
La soluzione basata su codec \emph{ibrido} comporta una notevole complessità
relativa alla fase di decodifica del video: spacchettizzazione dei GOP,
esclusione dell'audio e compensazione degli errori. Sarebbe necessaria una
ulteriore analisi sulla previsione interframe e interGOP utilizzata dai codec
layered: è necessario annullare tali informazioni poiché ogni descrizione
contiene solo alcune porzioni dei dati su cui tali statistiche sono state calcolate.
Sarebbe inevitabile, inoltre, la creazione di ulteriore ridondanza relativa
alla presenza del layer base in tutte le descrizioni MDC, oltre le informazioni
di arricchimento. \`E però indiscutibile l'eventuale enorme diffusione che avrebbe
tale codec in quanto permetterebbe l'immediata impiegabilità con la quasi totalità dei file video oggi esistenti.

\section{Codec audio}

\section{Rete distribuita}

\section{Regolazione della banda}

\section{Applicazioni destinate all'utente non esperto}